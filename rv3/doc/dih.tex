\documentclass{article}
\usepackage{graphicx}
\usepackage{amsmath}
\usepackage[margin=1.6in]{geometry}
\begin{document}

\section{Dihedral angle and its derivatives}

\subsection{Definition}

The dihedral $\phi$ of four points $i$, $j$, $k$ and $l$, 
is defined as the angle between the plane vectors 
$k$-$j$-$i$, and $l$-$k$-$j$, see Figure \ref{dihfig}.
%
More precisely,
\begin{equation}
  \cos \phi = \frac { \vec m \cdot \vec n} { m \  n},
  \label{eq:cosphi}
\end{equation}
%
where the plane vectors $\vec m$ and $\vec n$ are
\begin{equation}
\vec m = \vec r_{ij} \times \vec r_{kj}
\label{eq:m_ijk}
\end{equation}
%
\begin{equation}
\vec n = \vec r_{kj} \times \vec r_{kl} 
\label{eq:n_jkl}
\end{equation}
%
and $m = | \vec m |$ and $n = | \vec n |$.
%
Since both $\phi$ and $-\phi$ yield the same value of cosine,
we determine the sign of $\phi$ from
that of $\vec n \cdot \vec r_{ij}$;
the valid range of $\phi$ is $(-\pi, \pi)$. 
%
\begin{figure}[h]
\begin{center}
\includegraphics{dihfig.ps}
\caption{\label{dihfig}definition of a dihedral angle}
\end{center}
\end{figure}


\subsection{First order derivatives}


It is convenient to define 
\begin{eqnarray}
  \vec m_k  &=&   \frac {\vec m} { m^k }, \\
  \vec n_k  &=&   \frac {\vec n} { n^k },
\end{eqnarray}
where $k$ is $1, 2, \ldots$ (the $k=1$ case represents unit vectors). 
%
Eq. (\ref{eq:cosphi}) can be rewritten as 
$\cos \phi = \vec m_1 \cdot \vec n_1$,
and the derivative of $\phi$ can be computed as 
\begin{eqnarray*}
d \phi  &=&  -\frac{ d (\vec m_1 \cdot \vec n_1) } {\sin \phi} 
\end{eqnarray*}
where 
$\hat r_{kj} = \vec r_{kj}/|\vec r_{kj}|$,
and 
$\sin\phi = \vec m_1 \times \vec n_1 \cdot \hat r_{kj} $.

To derive derivatives of $m_1$ with respect to coordinates, 
we use Einstein notations.
Eq. (\ref{eq:m_ijk}) is equivalent to
$m_\alpha = \epsilon_{\alpha \beta \gamma} r_{ij \beta} r_{kj \gamma}$.
%
Thus, 
$\nabla_i \otimes \vec m$
is 
\begin{equation}
  \frac {\partial m_\alpha} {\partial r_{i \beta}} 
  = \epsilon_{\alpha \beta \gamma} r_{k j \gamma},
  \label{eq:dmadrib}
\end{equation}
where $\epsilon_{\alpha \beta \gamma}$ is the Levi-Civita symbol;
%
summation over repeated indices is assumed.
%
%or $\nabla_i \otimes \vec m = \times \vec r_{kj} \times$
Further,
$\nabla_i m$ is 
\begin{equation}
    \frac {\partial m} {\partial r_{i  \beta}} 
  = \frac {m_\alpha}{m} \frac {\partial m_\alpha} {\partial r_{i \beta}} 
  = \frac {\epsilon_{\alpha \beta \gamma} m_\alpha r_{k j \gamma}}{m}.
  \label{eq:dmdrib}
\end{equation}
%

We now show that
\begin{equation}
\nabla_i \otimes \vec m_1   
= \frac{ \nabla_i \otimes \vec m   } {m}
- \frac{ \nabla_i m \otimes \vec m } {m^2}   
= \vec m_2 \otimes (\vec m_1 \times \vec r_{kj}),
\label{eq:dim1}
\end{equation}
%
\begin{eqnarray*}
  \frac {\partial m_{\alpha}} {m \, \partial r_{i \beta} } 
  -
  \frac {\partial m} {\partial r_{i \beta}} 
  \frac {m_\alpha} {m^2} 
  &=&
  \frac{ \epsilon_{\alpha \beta \gamma} r_{kj \gamma} } { m }
  - \frac{ \epsilon_{\alpha' \beta \gamma} m_{\alpha'} r_{k j \gamma} m_\alpha }
    { m^3 } \\ 
  &=& 
  \left( 
       \delta_{\alpha \alpha'} 
       - \frac{ m_\alpha m_{\alpha'} }{ m^2 } 
  \right)  
  \frac{\epsilon_{\alpha' \beta \gamma} \, r_{kj \gamma} } {m} \\
  &=&
   \epsilon_{\alpha \mu \eta} \epsilon_{\alpha' \nu \eta}
   m_\mu m_\nu
  \frac{\epsilon_{\alpha' \beta \gamma} \, r_{kj \gamma} } {m^3} \\
  &=&
  \epsilon_{\alpha \mu \eta}   m_\mu m_\nu
  (\delta_{\nu \beta} \delta_{\eta \gamma} - 
   \delta_{\nu \gamma} \delta_{\eta \beta})
  \frac{ r_{kj \gamma} } {m^3} \\
  &=&
  (\epsilon_{\alpha \mu \gamma}   m_\mu m_\beta  - 
   \epsilon_{\alpha \mu \beta}   m_\mu m_\gamma )
  \frac{ r_{kj \gamma} } {m^3} \\
  &=&
  \frac{m_\beta \, \epsilon_{\alpha \mu \gamma}  m_\mu  r_{kj \gamma}} {m^3},
\end{eqnarray*}
%
where, on the third line we have used the identity 
$ \delta_{\alpha \alpha'} - s_\alpha s_{\alpha'} 
 = \epsilon_{\alpha \mu \eta} \epsilon_{\alpha' \nu \eta} s_\mu s_\nu $
for a unit vector $\vec s$; 
%
on the last line, the second term is dropped because 
$m_\gamma r_{kj \gamma}  = \vec m \cdot \vec r_{kj} = 0$.
%
Similarly, we have
\begin{eqnarray}
\nabla_j \otimes \vec m_1   &=&  -\vec m_2 \otimes (\vec m_1 \times \vec r_{ki}), \\
\nabla_k \otimes \vec m_1   &=&  -\vec m_2 \otimes (\vec m_1 \times \vec r_{ij}),
\end{eqnarray}
%
and for $n_1$,
\begin{eqnarray}
\nabla_j \otimes \vec n_1   &=&  -\vec n_2 \otimes (\vec n_1 \times \vec r_{kl}), \\
\nabla_k \otimes \vec n_1   &=&  -\vec n_2 \otimes (\vec n_1 \times \vec r_{lj}), \\
\nabla_l \otimes \vec n_1   &=&   \vec n_2 \otimes (\vec n_1 \times \vec r_{kj}),
\end{eqnarray}

The gradient $\phi$ with respect to coordinates are now readily obtained:
\begin{eqnarray*}
\nabla_i \phi &=& -\frac{ \nabla \otimes \vec m_1 \cdot \vec n_1  + m_1 \cdot \nabla \otimes \vec n_1 }
                  {\sin \phi} \\
              &=& -\frac{ \vec m_2 (\vec m_1 \times \vec r_{kj} \cdot \vec n_1) + 0 }
                  {\sin \phi} \\
              &=& \frac{ (\vec m_1 \times \vec n_1 \cdot \hat r_{kj}) \, r_{kj} }
                  {\sin \phi} \vec m_2 \\
              &=& r_{kj} \, \vec m_2,
\end{eqnarray*}

Other gradient terms can be obtained similarly, in summary:
\begin{eqnarray}
\nabla_i \phi   &=&  r_{kj} \, \vec m_2, \\
\nabla_j \phi   &=& -r_{kj} \, \vec m_2 + \vec S, \\
\nabla_k \phi   &=&  r_{kj} \, \vec n_2 - \vec S, \\
\nabla_l \phi   &=& -r_{kj} \, \vec n_2,
\end{eqnarray}
where
$\vec S = (\vec r_{ij} \cdot \hat r_{kj}) \ \vec m_2 
        + (\vec r_{kl} \cdot \hat r_{kj}) \ \vec n_2.$



\subsection{Conjugate field and its divergence}


The conjugate field $\vec u$ is defined as
\begin{equation}
\vec u = \frac {\nabla \phi} {\nabla \phi \cdot \nabla \phi},
\end{equation}
and its divergence
\begin{equation}
\nabla \cdot \vec u  = 
   \frac {\nabla^2 \phi} 
         {\nabla \phi \cdot \nabla \phi}
 - \frac {\nabla \phi \cdot \nabla \otimes \nabla \phi \cdot \nabla \phi} 
         {(\nabla \phi \cdot \nabla \phi)^2}.
\end{equation}
%
We thus need to compute 
$\nabla^2 \phi$
and
$\nabla \phi \cdot \nabla \otimes \nabla \phi \cdot \nabla \phi$.


\subsubsection{$\nabla^2 \phi$}
First, $\nabla \cdot \nabla_i \phi$ is easier,
%
\begin{eqnarray*}
\nabla_i^2 \phi   
&=& \nabla_i \cdot (r_{kj} \, \vec m_2) 
 =  r_{kj} \; ( \nabla_i \cdot \vec m_2 ) \\
&=& r_{kj} 
  \left(
    \frac {\nabla_i \cdot \vec m} {m^2} 
    - 2 \frac {\vec m \cdot \nabla_i m} {m^3}
  \right) = 0,
\end{eqnarray*}
where on the last line, we used
$\nabla_i \cdot \vec m = \vec m \cdot \nabla_i m = 0$,
from Eq. (\ref{eq:dmadrib}) and (\ref{eq:dmdrib}).
%
Similarly, since
$\nabla_j \cdot \vec m_2 = \nabla_k \cdot \vec m_2 = 0$,
and
$\nabla_j r_{jk} = \hat r_{jk}$,
%
$\nabla_j \cdot \nabla_i \phi = \nabla_k \cdot \nabla_i \phi = 0$.


Second, $\nabla \cdot \nabla_j \phi$, involves evaluating $\nabla \cdot \vec S$,
\begin{eqnarray*}
\nabla_k \cdot [ ( \vec r_{ij} \cdot \hat r_{kj}) \vec m_2]
&=& 
\nabla_k ( \vec r_{ij} \cdot \hat r_{kj}) \cdot \vec m_2
+
( \vec r_{ij} \cdot \hat r_{kj}) \; (\nabla_k \cdot \vec m_2) \\
&=& 
\vec m_2 \cdot \nabla_k \otimes \hat r_{kj} \cdot \vec r_{ij}
+ 0.
\end{eqnarray*}
%
Since,
$ \nabla_k \otimes \vec r_{kj} 
= \frac {\mathbf I} { r_{kj}}
- \frac { \nabla_k r_{kj} \otimes \vec r_{kj} }{ r_{kj}^2} 
= \frac { {\mathbf I - \hat r_{kj} \otimes \hat r_{kj}} } { r_{kj}},$
%
we have
\begin{eqnarray*}
\nabla_k \cdot [ ( \vec r_{ij} \cdot \hat r_{kj}) \vec m_2]
=
\vec m_2 \cdot \frac { {\mathbf I - \hat r_{kj} \otimes \hat r_{kj}} } { r_{kj}} \cdot \vec r_{ij} 
= 0 + 0 = 0.
\end{eqnarray*}

In summary, $\nabla^2 \phi = 0$.

\subsubsection{$\nabla \phi \cdot \nabla \otimes \nabla \phi \cdot \nabla \phi$}
We define 
$D_{ab} \equiv \nabla_a \phi \cdot \nabla_a \otimes \nabla_b \phi \cdot \nabla_b \phi$, 
where $a, b = i, j, k, l$.

First $D_{ii} = 0$, proof:
\begin{eqnarray*}
D_{ii} 
&=& \nabla_i \phi \cdot \nabla_i \otimes \nabla_i \phi \cdot \nabla_i \phi \\
&=& \nabla_i \phi \cdot r_{kj} \nabla_i \otimes \vec m_2 \cdot \nabla_i \phi \\
&=& \nabla_i \phi \cdot r_{kj} 
    \left(
    \frac {\nabla_i \otimes \vec m} {m^2}
    - 2 \frac{\nabla_i m \otimes \vec m}{m^3} 
    \right)
  \cdot \nabla_i \phi \\
&=& r_{kj} \left(
    \frac {\vec r_{kj} \cdot \nabla_i \phi \times \nabla_i \phi} {m^2}
    - 2 \frac{(\nabla_i \phi \cdot \nabla_i m) \; (\vec m \cdot \nabla_i \phi)}{m^3} 
    \right) \\
&=& 0 + 0 = 0
\end{eqnarray*}
where we have used a similar technique in deriving Eq. (\ref{eq:dim1}),
and $\nabla_i m = \vec r_{kj} \times \hat m$, 
and thus $\nabla_i \phi \cdot \nabla_i m = 0$.
We list the rest of nonzero items:

\begin{eqnarray}
D_{ij} 
&=& \frac {\sin \phi} {mn} \; \frac {r_{kj}^2} {m^2} \;
  z_{ki} \; z_{kl}, \\
%
D_{ik} 
&=&  \frac {\sin \phi} {mn} \; \frac {r_{kj}^2} {m^2} \; 
  z_{ij} \; z_{lj}, \\
%
D_{kl}
&=& \frac {\sin \phi} {mn} \; \frac {r_{kj}^2}{n^2} \; 
  z_{ij} \; z_{lj} , \\
%
D_{jl}
&=& \frac {\sin \phi} {mn} \; \frac {r_{kl}^2} {n^2} \;
  z_{ki} \; z_{kl} , \\
%
D_{jj} 
&=& - \frac{
    (\nabla_j \phi \otimes \nabla_j \phi)  : 
    (\vec r_{ij} \otimes \vec m_2 + \vec r_{kl} \otimes \vec n_2)
  } {r_{kj}} \nonumber \\
& & - 2 \frac {\sin \phi} {mn}  
  z_{ki} \; z_{kl} \;
  [
    (\nabla_j \phi \cdot \vec m_2) \; z_{ki}
   -(\nabla_j \phi \cdot \vec n_2) \; z_{kl}
  ], \\
%
D_{jk} 
&=&
\frac{ 
    (\nabla_j \phi \otimes \nabla_k \phi)  : 
    (\vec r_{ij} \otimes \vec m_2 + \vec r_{kl} \otimes \vec n_2)
} {r_{kj}} \nonumber \\
& & 
{} -\frac {\sin \phi} {mn}
  z_{ij} \; z_{ki} \;
  [
    (\nabla_j \phi \cdot \vec m_2) \; z_{lj}
   +(\nabla_k \phi \cdot \vec m_2) \; z_{kl}
  ] \nonumber \\
& &
 {} +\frac {\sin \phi} {mn}
  z_{lj} \; z_{kl} \;
  [
    (\nabla_j \phi \cdot \vec n_2) \; z_{kl}
   +(\nabla_k \phi \cdot \vec n_2) \; z_{ki}
  ], \\
%
D_{kk}
&=& - \frac{
    (\nabla_k \phi \otimes \nabla_k \phi)  : 
    (\vec r_{ij} \otimes \vec m_2 + \vec r_{kl} \otimes \vec n_2)
  } {r_{kj}} \nonumber \\
& & {}- 2 \frac {\sin \phi} {mn}  
  z_{lj} \; z_{ij} \;
  [
    (\nabla_k \phi \cdot \vec m_2) \; z_{ij}
   -(\nabla_k \phi \cdot \vec n_2) \; z_{lj}
  ],
\end{eqnarray}
where, component parallel to $\hat r_{kj}$ is denoted by $z$, e.g., 
$z_{ij} \equiv \vec r_{ij} \cdot \hat r_{kj}$.
Note, although not obvious, the expression for $D_{jk}$ is symmetrical to $j$ and $k$. 
\begin{align*}
&\ \ \ (\nabla_j \phi \otimes \nabla_k \phi) : (\vec r_{ij} \otimes \vec m_2)
+ (\nabla_j \phi \otimes \nabla_k \phi) : (\vec r_{kl} \otimes \vec n_2) \\
&= (\nabla_j \phi \cdot \vec r_{ij})
  (\nabla_k \phi \cdot \vec m_2)
+ (\nabla_j \phi \cdot \vec r_{kl})
  (\nabla_k \phi \cdot \vec n_2) \\
&= z_{kl} \; (\vec n_2 \cdot \vec r_{ij})
  (\nabla_k \phi \cdot \vec m_2 ) 
+  z_{ik} \; (\vec m_2 \cdot \vec r_{kl})
  (\nabla_k \phi \cdot \vec n_2 ) \\
&= -z_{kl} \; (\vec n_2 \cdot \vec r_{ij})
   (z_{ij} \vec m_2 + z_{lj} \vec n_2) \cdot \vec m_2 
-  z_{ik} \; (\vec m_2 \cdot \vec r_{kl})
   (r_{ij} \vec m_2 + r_{lj} \vec n_2) \cdot \vec n_2 \\
&= - \; \frac{V}{n^2}
  \left( 
    \frac {z_{kl} z_{ij}}{m^2} +
    z_{kl} z_{lj} \; \vec m_2 \cdot \vec n_2 
  \right)
- \; \frac{V}{m^2}
  \left( 
    z_{ik} z_{ij} \; \vec m_2 \cdot \vec n_2 +
    \frac {z_{ik} z_{lj}} {n^2} 
  \right) \\  
&= - \frac{V}{m^2 n^2} 
  \left(
    z_{kl} z_{ij} + z_{ik} z_{lj}
  \right)
   - V \vec m_2 \cdot \vec n_2
   \left(
     \frac {z_{kl} z_{lj}} {n^2} 
    +\frac {z_{ik} z_{ij}} {m^2} 
   \right).
\end{align*}
%


\end{document}
