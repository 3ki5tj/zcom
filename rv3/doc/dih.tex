\documentclass{article}
\usepackage{graphicx}
\usepackage{amsmath}
\usepackage[margin=1.6in]{geometry}
\begin{document}

\newcommand{\vct}[1]{\mathbf{#1}}
\newcommand{\vx}{\vct{x}}
\newcommand{\vy}{\vct{y}}
\newcommand{\vz}{\vct{z}}
\newcommand{\vm}{\vct{m}}
\newcommand{\vn}{\vct{n}}
\newcommand{\vr}{\vct{r}}
\newcommand{\vxh}{\hat{\vct{x}}}
\newcommand{\vyh}{\hat{\vct{y}}}
\newcommand{\vzh}{\hat{\vct{z}}}
\newcommand{\vmh}{\hat{\vct{m}}}
\newcommand{\vnh}{\hat{\vct{n}}}
\newcommand{\vrh}{\hat{\vct{r}}}
\newcommand{\vmhh}{\hat{\hat{\vct{m}}}}
\newcommand{\vnhh}{\hat{\hat{\vct{n}}}}
\newcommand{\diphi}{\nabla_i \phi}
\newcommand{\djphi}{\nabla_j \phi}
\newcommand{\dkphi}{\nabla_k \phi}
\newcommand{\dlphi}{\nabla_l \phi}

\section{Dihedral angle and its derivatives}

\subsection{Definition}

The dihedral $\phi$ of four points $i$, $j$, $k$ and $l$, 
is defined as the angle between the two planes 
$k$-$j$-$i$, and $l$-$k$-$j$, see Fig. \ref{dihfig}.
%
We define the two respective plane vectors $\vm$ and $\vn$ as
\begin{equation}
\vm = \vr_{ij} \times \vr_{kj}
\label{eq:m}
\end{equation}
%
\begin{equation}
\vn = \vr_{kj} \times \vr_{kl} 
\label{eq:n}
\end{equation}
%
The dihedral can then be computed from,
\begin{equation}
  \cos \phi = \vmh \cdot \vnh
  \label{eq:cosphi}
\end{equation}
%
where $\vmh = \vm/m$, $\vnh = \vn/n$
%
and $m = | \vm |$, $n = | \vn |$.
%
Since both $\phi$ and $-\phi$ yield the same value of cosine,
we determine the sign of $\phi$ from that of $\vn \cdot \vr_{ij}$;
the valid range of $\phi$ is $(-\pi, \pi)$.
Thus we also have 
\begin{equation}
  \vmh \times \vnh  = \sin \phi \, \vrh_{kj}
  \label{eq:sinphi}
\end{equation}

%
\begin{figure}[h]
\begin{center}
\includegraphics{dihfig.ps}
\caption{\label{dihfig}The definition of a dihedral}
\end{center}
\end{figure}

For later convenience, we further introduce
$\vmhh = \vm/m^2$ and $\vnhh = \vn/n^2$.




\subsection{Gradient}

\subsubsection{Geometric derivation}

In the follows, we shall derive the gradient of the dihedral geometrically.
%
The key is to observe that $\phi$ changes only via the unit plane vectors $\vmh$ and $\vnh$,
i.e., any change that leaves the two unit vectors invariant has no effect to 
the dihedral $\phi$.
This observation helps to deduce the direction of the gradient.

We derive $\diphi$ based on two observations: 
1) the position of $i$ affects vector $\vmh$, but not $\vnh$;
2) any change of $\vr_i$ within the $i$-$j$-$k$ plane leaves $\vmh$ invariant.
%
Thus the gradient $\diphi$ must be along the direction that 
    drives $i$ to move out of the $i$-$j$-$k$ plane,
i.e., $\diphi$ must be parallel to $\vmh$ or 
\[
\diphi  = a \, \vmh,
\]
where $a$ is a constant to be determined.
%
The sign of $a$ should be positive according to Fig. \ref{dihfig} (i.e., moving along $\vmh$ the dihedral increases).
%
The amplitude of $a$ is determined as the follows.
Consider rotating the point $i$ (holding $j$, $k$, and $l$) around 
the line connecting $j$ and $k$, whose direction can be represented by the unit vector $\vrh_{kj}$ or $\vzh$,
by $-\delta \phi$, which in turn increases the dihedral by $\delta \phi$.
The rotation displaces $\vr_i$  by
$\delta \vr_i = \vrh_{kj} \times \vr_{ij} \, (-\delta \phi)$.
Compare with Eq. (\ref{eq:m}), we found 
\[
 \delta \vr_i =  (m/r_{kj}) \, \delta \phi \, \vmh.
\]
Geometrically, $m/r_{jk}$ is equal to twice the area of the triangle $i$-$j$-$k$
 divided by the side length $r_{jk}$,
or the vertical distance from $i$ to the side $j$-$k$.
%
%
Since the change of the dihedral can be computed from the gradient as 
$\delta \phi = \delta \vr_i \cdot \diphi $, we have 
$a = r_{kj}/m$, and 
\[
\diphi = (r_{kj}/m) \, \vmh = r_{kj} \, \vmhh.
\]

By symmetry, we have
\[
\dlphi = -(r_{kj}/n) \, \vnh = -r_{kj} \, \vnhh.
\]
Note the change of the sign due to that 
the vector $\vnh$ is along a direction that decreases the dihedral.

To derive $\djphi$, we note that a displacement along $\vrh_{kj}$ 
leaves the dihedral invariant. 
Accordingly $\djphi$ must be perpendicular to $\vrh_{kj}$ 
and thus can be expressed as a linear combination of $\vmh$ and $\vnh$, i.e.,
\[
\djphi = b \, \vmh + c \, \vnh.
\]

To derive $c$, we rotate $j$ (holding $i$, $k$ and $l$) around 
an axis parallel to $\vmh$ (or $-\vyh$) and passing through $k$ by $\delta \theta$.
Such a rotation displaces $\vr_j$ by 
%
$\delta \vr_j = r_{kj} \, \delta \theta \, \vxh$,
%
and produces a change of the dihedral by
\begin{equation}
\delta \phi
  = \djphi \cdot \delta \vr_j 
  = c \, r_{kj} \, \sin \phi \, \delta \theta, 
\label{eq:dphij1}
\end{equation}
where we have used the fact $\vmh \cdot \vxh = 0$, $\vnh \cdot \vxh = \sin \phi$.


The rotation however makes $\vr_{kj}$ deviate from the original $z$ axis,
  and makes the computation of $\phi$ inconvenient.
We therefore rotate back the \emph{entire} system around the same axis by $-\delta \theta$.
Note however the second rotation does not change the dihedral $\phi$.
The overall effect of the two rotations is that 
  $i$ and $l$ are rotated around the axis by $-\delta \theta$, 
   while $j$ and $k$ remain unchanged.
%
%
Further, the rotation leaves $\vr_i$ in the $i$-$j$-$k$ plane,
so the change of $\vr_i$ does not affect the dihedral.
%
Thus $\delta \phi$ can be computed as 
\begin{equation*}
\begin{split}
\delta \phi
  &= \dlphi \cdot \delta \vr_l = 
 \frac{-r_{kj} \vnh}{n} \cdot (-\vmh \times \vr_{lk}) \, \delta \theta  \\
  &= \frac{r_{kj}}{n} \, \vr_{lk} \cdot (\vnh \times \vmh) \, \delta \theta
  = -(\vr_{lk} \cdot \vrh_{kj} / n )\, r_{kj} \, \sin \phi \, \delta \theta,
\end{split}
\end{equation*}
%
where we have used Eq. (\ref{eq:sinphi}) in the last step.
%
Compare with Eq. (\ref{eq:dphij1}), we find $c=-\vr_{lk}\cdot\vrh_{kj}/n$.


Similarly, to find $b$, we first rotate $j$ around the axis passing through $k$ 
and parallel to $\vnh$ by $\delta \theta'$,
then rotate the entire system in the other direction by $-\delta \theta'$.
%
The change of the dihedral can again be derived in two ways:
\begin{equation*}
\begin{split}
\delta \phi'
  & = \dlphi \cdot \delta \vr_l = -b \, r_{kj} \sin \phi \, \delta \theta' \\
  & = \diphi \cdot \delta \vr_i
    = \frac{r_{kj} \vmh} {m} \cdot (-\vnh \times \vr_{ik}) \, \delta \theta'
    = - (\vr_{ik} \cdot \vrh_{kj}/m) \, r_{k j} \,  \sin \phi \,  \delta \theta'
\end{split}
\end{equation*}
Thus $b = \vr_{ik} \cdot \vrh_{kj}/m$, and 
%
\[
  \djphi = \vr_{ik} \cdot \vrh_{kj} \, \vmhh - \vr_{lk}\cdot \vrh_{kj} \, \vnhh.
\]
$\dkphi$ can be obtained by symmetry.

In summary, we have:
\begin{equation}
\begin{split}
\diphi   &=  r_{kj} \, \vmhh  \\
\djphi   &=  (\vr_{ik} \cdot \vrh_{kj}) \, \vmhh - (\vr_{lk} \cdot \vrh_{kj}) \, \vnhh 
         = -r_{kj} \, \vmhh + \vct S \\
\dkphi   &=  (\vr_{lj} \cdot \vrh_{kj}) \, \vnhh - (\vr_{ij} \cdot \vrh_{kj}) \, \vmhh
         =  r_{kj} \, \vnhh - \vct S \\
\dlphi   &= -r_{kj} \, \vnhh,
\end{split}
\label{eq:gradall}
\end{equation}
where $\vct S \equiv (\vr_{ij}\cdot \vrh_{kj}) \vmhh + (\vr_{kl} \cdot \vrh_{kj}) \, \vnhh$.
We observe $\diphi + \djphi + \dkphi + \dlphi = 0$ as it should be, 
because the dihedral is an invariant under an arbitrary global translation.




\subsubsection{Derivation using vectors and tensors}

From Eq. (\ref{eq:cosphi})
\begin{eqnarray*}
d \phi  &=&  -\frac{ (d\,\vmh) \cdot \vnh + \vmh \cdot (d\,\vnh) } {\sin \phi} 
\end{eqnarray*}
where 
$\vrh_{kj} = \vr_{kj}/r_{kj}$, with $r_{kj} = |\vr_{kj}|$,
and 
$\sin\phi = \vmh \times \vnh \cdot \vrh_{kj} $.

To derive derivatives of $\vmh$ with respect to coordinates, 
we use notations in tensor analysis (but treat contravariant and covariant vectors as the same).
Eq. (\ref{eq:m}) is equivalent to
$m_\alpha = \epsilon_{\alpha \beta \gamma} r_{ij \beta} r_{kj \gamma}$.
%
Thus, 
$\nabla_i \otimes \vm$
is 
\begin{equation}
  \frac {\partial m_\alpha} {\partial r_{i \beta}} 
  = \epsilon_{\alpha \beta \gamma} r_{k j \gamma},
  \label{eq:dmadrib}
\end{equation}
where $\epsilon_{\alpha \beta \gamma}$ is the Levi-Civita symbol;
%
summation over repeated indices is assumed.
%
Further,
$\nabla_i m$ is 
\begin{equation}
    \frac {\partial m} {\partial r_{i  \beta}} 
  = \frac {m_\alpha}{m} \frac {\partial m_\alpha} {\partial r_{i \beta}} 
  = \frac {\epsilon_{\alpha \beta \gamma} m_\alpha r_{k j \gamma}}{m}.
  \label{eq:dmdrib}
\end{equation}
%

We now show that
\begin{equation*}
\nabla_i \otimes \vmh   
= \frac{ \nabla_i \otimes \vm   } {m}
- \frac{ \nabla_i m \otimes \vm } {m^2}   
= \vmhh \otimes (\vmh \times \vr_{kj}),
\end{equation*}
%
\begin{equation*}
\begin{split}
  \frac {\partial (m_{\alpha} / m)} {\partial r_{i \beta} } 
  &=
  \frac {\partial m_{\alpha}} {m \, \partial r_{i \beta} } 
  -
  \frac {\partial m} {\partial r_{i \beta}} 
  \frac {m_\alpha} {m^2}
  =
  \frac{ \epsilon_{\alpha \beta \gamma} r_{kj \gamma} } { m }
  - \frac{ \epsilon_{\alpha' \beta \gamma} m_{\alpha'} r_{k j \gamma} m_\alpha }
    { m^3 } \\ 
  &= 
  \left( 
       \delta_{\alpha \alpha'} 
       - \frac{ m_\alpha m_{\alpha'} }{ m^2 } 
  \right)  
  \frac{\epsilon_{\alpha' \beta \gamma} \, r_{kj \gamma} } {m} \\
  &=
   \epsilon_{\alpha \mu \eta} \epsilon_{\alpha' \nu \eta}
   m_\mu m_\nu
  \frac{\epsilon_{\alpha' \beta \gamma} \, r_{kj \gamma} } {m^3} \\
  &=
  \epsilon_{\alpha \mu \eta}   m_\mu m_\nu
  (\delta_{\nu \beta} \delta_{\eta \gamma} - 
   \delta_{\nu \gamma} \delta_{\eta \beta})
  \frac{ r_{kj \gamma} } {m^3} \\
  &=
  (\epsilon_{\alpha \mu \gamma}   m_\mu m_\beta  - 
   \epsilon_{\alpha \mu \beta}   m_\mu m_\gamma )
  \frac{ r_{kj \gamma} } {m^3} \\
  &=
  \frac{m_\beta \, \epsilon_{\alpha \mu \gamma}  m_\mu  r_{kj \gamma}} {m^3},
\end{split}
\end{equation*}
%
where, on the third line we have used the identity 
$ \delta_{\alpha \alpha'} - s_\alpha s_{\alpha'} 
 = \epsilon_{\alpha \mu \eta} \epsilon_{\alpha' \nu \eta} s_\mu s_\nu $
for a unit vector $\vct s$; 
%
on the last line, the second term is dropped because 
$m_\gamma r_{kj \gamma}  = \vm \cdot \vr_{kj} = 0$.
%
%
%
We thus have
%
\begin{equation}
\begin{split}
\nabla_i \otimes \vmh   &=  \vmhh \otimes (\vmh \times \vr_{kj}) \\
\nabla_j \otimes \vmh   &=  -\vmhh \otimes (\vmh \times \vr_{ki}) \\
\nabla_k \otimes \vmh   &=  -\vmhh \otimes (\vmh \times \vr_{ij}),
\end{split}
\label{eq:dim1}
\end{equation}
where the last two equations follow from symmetry.  Similarly we have
%
%
\begin{equation}
\begin{split}
\nabla_j \otimes \vnh   &=  -\vnhh \otimes (\vnh \times \vr_{kl}), \\
\nabla_k \otimes \vnh   &=  -\vnhh \otimes (\vnh \times \vr_{lj}), \\
\nabla_l \otimes \vnh   &=   \vnhh \otimes (\vnh \times \vr_{kj}),
\end{split}
\end{equation}

The gradient $\phi$ with respect to coordinates are now readily obtained:
\begin{eqnarray*}
\diphi &=& -\frac{ \nabla_i \otimes \vmh \cdot \vnh  + \vmh \cdot \nabla_i \otimes \vnh }
                  {\sin \phi} \\
              &=& -\frac{ \vmhh (\vmh \times \vr_{kj} \cdot \vnh) + 0 }
                  {\sin \phi} \\
              &=& \frac{ (\vmh \times \vnh \cdot \vrh_{kj}) \, r_{kj} }
                  {\sin \phi} \vmhh \\
              &=& r_{kj} \, \vmhh,
\end{eqnarray*}

Other gradient terms can be obtained similarly, the results are the same as Eqs. (\ref{eq:gradall}).




\subsection{Conjugate field and its divergence}


The conjugate field $\vct{v}$ is defined as
\begin{equation}
\vct{v} = \frac {\nabla \phi} {\nabla \phi \cdot \nabla \phi},
\end{equation}
and its divergence
\begin{equation}
\nabla \cdot \vct{v}  = 
   \frac {\nabla^2 \phi} 
         {\nabla \phi \cdot \nabla \phi}
 - \frac {\nabla \phi \cdot \nabla \otimes \nabla \phi \cdot \nabla \phi} 
         {(\nabla \phi \cdot \nabla \phi)^2}.
\label{eq:diverg}
\end{equation}
%
We thus need to compute 
$\nabla^2 \phi$
and
$\nabla \phi \cdot \nabla \otimes \nabla \phi \cdot \nabla \phi$.


As the operator $\nabla$ can be any one of the 
$\nabla_i$, $\nabla_j$, $\nabla_k$ and $\nabla_l$,
we have to compute the following combinations 
$i$-$i$, $i$-$j$, $i$-$k$, $i$-$l$ and $j$-$k$.
%
The others can be deduced from the symmetry, 
e.g., 
$j$-$i$ is equivalent to $i$-$j$, and
$j$-$l$ can be deduced from $i$-$k$.

In the follows, we shall compute the five combinations for the two terms in 
Eq. (\ref{eq:diverg}).

\subsubsection{$\nabla_a \cdot \nabla_b \phi$}

We shall show that every combination
\begin{equation} 
\nabla_a \cdot \nabla_b \phi = 0,
\end{equation}
for any $a, b = i, j, k, l$.

First, the easiest: $\nabla_l \cdot \diphi = 0$
because $\diphi$ is independent of the coordinates of $l$.


Second,
\begin{eqnarray*}
\nabla_i^2 \phi   
&=& \nabla_i \cdot (r_{kj} \, \vmhh) 
 =  r_{kj} \; ( \nabla_i \cdot \vmhh ) \\
&=& r_{kj} 
  \left(
    \frac {\nabla_i \cdot \vm} {m^2} 
    - \frac {2 \vm \cdot \nabla_i m} {m^3}
  \right) = 0,
\end{eqnarray*}
where on the last line, we used
$\nabla_i \cdot \vm = \vm \cdot \nabla_i m = 0$,
from Eq. (\ref{eq:dmadrib}) and (\ref{eq:dmdrib}).


Similarly, since
$\nabla_j \cdot \vmhh = \nabla_k \cdot \vmhh = 0$,
and
$\nabla_j r_{jk} = \vrh_{jk}$,
%
$\nabla_j \cdot \diphi = \nabla_k \cdot \diphi = 0$.
%

Last, $\nabla \cdot \djphi$, involves evaluating $\nabla \cdot \vct S$,
\begin{eqnarray*}
\nabla_k \cdot [ ( \vr_{ij} \cdot \vrh_{kj}) \, \vmhh]
&=& 
\nabla_k ( \vr_{ij} \cdot \vrh_{kj}) \cdot \vmhh
+
( \vr_{ij} \cdot \vrh_{kj}) \; (\nabla_k \cdot \vmhh) \\
&=& 
\vmhh \cdot \nabla_k \otimes \vrh_{kj} \cdot \vr_{ij}
+ 0.
\end{eqnarray*}
%
Since,
$ \nabla_k \otimes \vr_{kj} 
= \frac {\mathbf I} { r_{kj}}
- \frac { \nabla_k r_{kj} \otimes \vr_{kj} }{ r_{kj}^2} 
= \frac { {\mathbf I - \vrh_{kj} \otimes \vrh_{kj}} } { r_{kj}},$
%
we have
\begin{eqnarray*}
\nabla_k \cdot [ ( \vr_{ij} \cdot \vrh_{kj}) \, \vmhh]
=
\vmhh \cdot \frac { {\mathbf I - \vrh_{kj} \otimes \vrh_{kj}} } { r_{kj}} \cdot \vr_{ij} 
= 0 + 0 = 0.
\end{eqnarray*}



\subsubsection{$\nabla_a \phi \cdot \nabla_a \otimes \nabla_b \phi \cdot \nabla_b \phi$}
We define 
$D_{ab} \equiv \nabla_a \phi \cdot \nabla_a \otimes \nabla_b \phi \cdot \nabla_b \phi$, 
where $a, b = i, j, k, l$.

First $D_{ii} = 0$, proof:
\begin{eqnarray*}
D_{ii} 
&=& \diphi \cdot \nabla_i \otimes \diphi \cdot \diphi \\
&=& \diphi \cdot r_{kj} \nabla_i \otimes \vmhh \cdot \diphi \\
&=& \diphi \cdot r_{kj} 
    \left(
    \frac {\nabla_i \otimes \vm} {m^2}
    - 2 \frac{\nabla_i m \otimes \vm}{m^3} 
    \right)
  \cdot \diphi \\
&=& r_{kj} \left(
    \frac {\vr_{kj} \cdot \diphi \times \diphi} {m^2}
    - 2 \frac{(\diphi \cdot \nabla_i m) \; (\vm \cdot \diphi)}{m^3} 
    \right) \\
&=& 0 + 0 = 0
\end{eqnarray*}
where we have used a similar technique in deriving Eq. (\ref{eq:dim1}),
and $\nabla_i m = \vr_{kj} \times \vmh$, 
and thus $\diphi \cdot \nabla_i m = 0$.
We list the nonzero items:

\begin{equation}
\begin{split}
D_{ij} &= s \, z_{ki} \, z_{kl} \; r_{kj}^2/m^2   \\
%
D_{ik} &= s \, z_{ij} \, z_{lj} \; r_{kj}^2/m^2 \\
%
D_{kl} &= s \, z_{ij} \, z_{lj} \; r_{kj}^2 / n^2 \\
%
D_{jl} &= s \, z_{ki} \, z_{kl} \; r_{kl}^2/n^2 \\
%
D_{jj} 
&= - (\djphi \otimes \djphi)  : 
    (\vr_{ij} \otimes \vmhh + \vr_{kl} \otimes \vnhh)
  / r_{kj} \\
&  - 2 s \, 
  z_{ki} \; z_{kl} \;
  [
    (\djphi \cdot \vmhh) \; z_{ki}
   -(\djphi \cdot \vnhh) \; z_{kl}
  ], \\
%
D_{jk} 
&= (\djphi \otimes \dkphi)  : 
    (\vr_{ij} \otimes \vmhh + \vr_{kl} \otimes \vnhh)
 / r_{kj}  \\
&  
 - s \,
  z_{ij} \; z_{ki} \;
  [
    (\djphi \cdot \vmhh) \; z_{lj}
   +(\dkphi \cdot \vmhh) \; z_{kl}
  ] \nonumber \\
& 
 +s \,
  z_{lj} \; z_{kl} \;
  [
    (\djphi \cdot \vnhh) \; z_{kl}
   +(\dkphi \cdot \vnhh) \; z_{ki}
  ], \\
%
D_{kk}
&= - 
    (\dkphi \otimes \dkphi)  : 
    (\vr_{ij} \otimes \vmhh + \vr_{kl} \otimes \vnhh)
   /r_{kj} \\
&  - 2 s \,
  z_{lj} \; z_{ij} \;
  [ \
    (\dkphi \cdot \vmhh) \; z_{ij} \
   -(\dkphi \cdot \vnhh) \; z_{lj} \
  ],
\end{split}
\end{equation}
where, component parallel to $\vrh_{kj}$ is denoted by $z$, e.g., 
$z_{ij} \equiv \vr_{ij} \cdot \vrh_{kj}$, 
%
$s \equiv \frac {\sin \phi} {mn}$.
Note, although not obvious, the expression for $D_{jk}$ is symmetrical to $j$ and $k$. 
\begin{align*}
&\ \ \ (\djphi \otimes \dkphi) : (\vr_{ij} \otimes \vmhh)
+ (\djphi \otimes \dkphi) : (\vr_{kl} \otimes \vnhh) \\
&= (\djphi \cdot \vr_{ij})
  (\dkphi \cdot \vmhh)
+ (\djphi \cdot \vr_{kl})
  (\dkphi \cdot \vnhh) \\
&= z_{kl} \; (\vnhh \cdot \vr_{ij})
  (\dkphi \cdot \vmhh ) 
+  z_{ik} \; (\vmhh \cdot \vr_{kl})
  (\dkphi \cdot \vnhh ) \\
&= -z_{kl} \; (\vnhh \cdot \vr_{ij})
   (z_{ij} \vmhh + z_{lj} \vnhh) \cdot \vmhh 
-  z_{ik} \; (\vmhh \cdot \vr_{kl})
   (r_{ij} \vmhh + r_{lj} \vnhh) \cdot \vnhh \\
&= - \; \frac{V}{n^2}
  \left( 
    \frac {z_{kl} z_{ij}}{m^2} +
    z_{kl} z_{lj} \; \vmhh \cdot \vnhh 
  \right)
- \; \frac{V}{m^2}
  \left( 
    z_{ik} z_{ij} \; \vmhh \cdot \vnhh +
    \frac {z_{ik} z_{lj}} {n^2} 
  \right) \\  
&= - \frac{V}{m^2 n^2} 
  \left(
    z_{kl} z_{ij} + z_{ik} z_{lj}
  \right)
   - V \vmhh \cdot \vnhh
   \left(
     \frac {z_{kl} z_{lj}} {n^2} 
    +\frac {z_{ik} z_{ij}} {m^2} 
   \right).
\end{align*}
%


\end{document}
