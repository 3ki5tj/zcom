\documentclass{article}
\usepackage{amsmath}
\begin{document}



\section{Derivatives of the potential energy distribution $\rho(U)$}
%
%
%
For a canonical ensemble, $\rho(U)$,
\begin{align}
\frac {\partial \ln \rho} 
      {\partial U}
    & = \langle \nabla \cdot \mathbf v \rangle_U - \beta, \\
\frac {\partial^2 \ln \rho}
      {\partial U^2}
    & = \langle \mathbf v \cdot \nabla (\nabla \cdot \mathbf v) \rangle_U
      + \langle \Delta (\nabla \cdot \mathbf v)^2 \rangle_U.
\end{align}
where $\mathbf v = \nabla U/(\nabla U \cdot \nabla U)$.






\subsection{Explicit formulas}
%
%
%
The divergence term $\nabla \cdot \mathbf v$ can be computed as the follows
%
\begin{align*}
\nabla \cdot \mathbf v
     = \frac { \nabla^2 U }
              { \nabla U \cdot \nabla U }
      - \frac { 2 \nabla U \cdot \nabla \nabla U \cdot \nabla U }
              { (\nabla U \cdot \nabla U)^2 }
     = \frac { L }
              { \mathbf g^2 }
      - \frac { M }
              { (\mathbf g^2)^2 },
\end{align*}
%
where we have defined
the gradient 
$\mathbf g \equiv \nabla U$, 
the Laplacian 
$L \equiv \nabla^2 U$,
and
$M \equiv 2 \mathbf g \cdot \nabla \nabla U \cdot \mathbf g$;
The notation $\mathbf g^2$ is short for $\mathbf g \cdot \mathbf g$.


The second derivative $\mathbf v \cdot \nabla (\nabla \cdot \mathbf v)$
\begin{align*}
\mathbf v \cdot \nabla (\nabla \cdot \mathbf v)
    = &  \frac { \mathbf v \cdot \nabla L }
              { \mathbf g^2 }
      - \frac { L (2 \mathbf v \cdot \nabla \nabla U \cdot \mathbf g) }
              { (\mathbf g^2)^2 }
     - \frac { \mathbf v \cdot \nabla M }
              { (\mathbf g^2)^2 } 
      + \frac { M (2 \mathbf v \cdot 2 \nabla \nabla U \cdot \mathbf g)}
              { (\mathbf g^2)^3 } 
\\
   = & \frac { \mathbf g \cdot \nabla L }
              { (\mathbf g^2)^2 }
      - \frac { L M }
              { (\mathbf g^2)^3 }
     - \frac { 4 \mathbf g \cdot \nabla \nabla U \cdot \nabla \nabla U \cdot \mathbf g }
              { (\mathbf g^2)^3 } 
      - \frac { 2 \mathbf g \mathbf g : \nabla \nabla \nabla U \cdot \mathbf g }
              { (\mathbf g^2)^3 } 
      + \frac { 2 M^2}
              { (\mathbf g^2)^4 } 
\\
   = & \frac { \mathbf g \cdot \nabla L }
              { (\mathbf g^2)^2 }
     - \frac { L M + \mathbf h^2  + 2 \mathbf g \mathbf g : \nabla \nabla \nabla U \cdot \mathbf g}
              { (\mathbf g^2)^3 } 
      + \frac { 2 M^2}
              { (\mathbf g^2)^4 },
\end{align*}
where we defined $\mathbf h \equiv 2 \mathbf g \cdot \nabla \nabla U$.

Lastly, we can define a virial-like quantity for the divergence (instead of the potential energy) as 
$\mathbf r \cdot \nabla (\nabla \cdot \mathbf v)$, which can be evaluated as
\begin{align*}
\mathbf r \cdot \nabla (\nabla \cdot \mathbf v)
    = &  \frac { \mathbf r \cdot \nabla L }
              { \mathbf g^2 }
      - \frac { L (2 \mathbf r \cdot \nabla \nabla U \cdot \mathbf g) }
              { (\mathbf g^2)^2 }
     - \frac { \mathbf r \cdot \nabla M }
              { (\mathbf g^2)^2 } 
      + \frac { M (2 \mathbf r \cdot 2 \nabla \nabla U \cdot \mathbf g)}
              { (\mathbf g^2)^3 } 
\\
   = & \frac { \mathbf r \cdot \nabla L }
              { \mathbf g^2 }
      - \frac { L (\mathbf r \cdot \mathbf h) }
              { (\mathbf g^2)^2 }
     - \frac { \mathbf h \cdot \mathbf s }
              { (\mathbf g^2)^2 } 
      - \frac { 2 \mathbf g \mathbf g : \nabla \nabla \nabla U \cdot \mathbf r }
              { (\mathbf g^2)^2 } 
      + \frac { 2 M (\mathbf r \cdot \mathbf h)}
              { (\mathbf g^2)^3 } 
\\
   = & \frac { \mathbf r \cdot \nabla L }
              { \mathbf g^2 }
     - \frac { L (\mathbf r \cdot \mathbf h) + \mathbf h \cdot \mathbf s + 2 \mathbf g \mathbf g : \nabla \nabla \nabla U \cdot \mathbf r}
              { (\mathbf g^2)^2 }
      + \frac { 2 M (\mathbf r \cdot \mathbf h)}
              { (\mathbf g^2)^3 },
\end{align*}
where $\mathbf s = 2 \mathbf r \cdot \nabla \nabla U$.



\subsection{Compute quantities in terms of a pair potential}

The following definitions are helpful,
\begin{align*}
  \phi(r) & = u'(r)/r \\
  \psi(r) & = \phi'(r)/r \\
  \xi(r)  & = \psi'(r)/r.
\end{align*}



\subsubsection{The Laplacian $L$}
\begin{align*}
L = \nabla^2 U 
  = \sum_k \nabla_k^2 U
  = \sum_{(i, j)} \sum_k \nabla_k^2 u(r_{ij})
  = \sum_{(i, j)} (\nabla_i^2 u(r_{ij}) + \nabla_j^2 u(r_{ij}))
\end{align*}
To expand, 
\begin{align*}
\nabla_i u(r_{ij}) 
& = \phi(r_{ij}) \, \mathbf r_{ij}, \\
\nabla_i^2 u(r_{ij})
& = \nabla_i \phi(r_{ij}) \cdot \mathbf r_{ij} + d \phi(r_{ij})
= \psi(r_{ij}) \, r_{ij}^2  + d \phi(r_{ij}), \\
L = \sum_{(i, j)} L_{ij} & = \sum_{(i, j)} 2 [ \psi(r_{ij}) r_{ij}^2 + d \phi(r_{ij}) ].
\end{align*}
$d = 3$ for the 3D case.



\subsubsection{$\mathbf g \cdot \nabla L$}

\begin{align*}
\nabla_k L = \delta_{ik} \nabla_i L_{ij} + \delta_{jk} \nabla_j L_{ij}. 
\end{align*}
where $\nabla_i L_{ij} = 2 [ \xi(r_{ij}) r_{ij}^2 + 2 \psi(r_{ij}) + d \psi(r_{ij}) ] \mathbf r_{ij}$.
\begin{align*}
\mathbf g \cdot \nabla L
= \sum_k \mathbf g_k \cdot \nabla_k L
= \sum_{(i, j)} 2 [ \xi(r_{ij}) r_{ij}^2 + (2 + d) \psi(r_{ij}) ] (\mathbf r_{ij} \cdot \mathbf g_{ij}),
\end{align*}
where $\mathbf g_{ij} = \mathbf g_i - \mathbf g_j$.



\subsubsection{$\mathbf h = 2 \mathbf g \cdot \nabla \nabla U$}

\begin{align*}
\mathbf h_k &= 2 \mathbf g \cdot \nabla \nabla U
  = 2 \sum_l \mathbf g_l \cdot \nabla_k \nabla_l U
  = 2 \sum_{(i, j)} \sum_l \mathbf g_l \cdot \nabla_k \nabla_l u(r_{ij})
\\
  &= 2 \sum_{(i, j)} \sum_l \mathbf g_l \cdot \nabla_k 
    (\delta_{il} - \delta_{jl}) \, \phi(r_{ij}) \mathbf r_{ij} 
  = 2 \sum_{(i, j)} \mathbf g_{ij} \cdot \nabla_k 
    \, \phi(r_{ij}) \mathbf r_{ij}
\\
  &= 2 \sum_{(i, j)} \mathbf g_{ij} \cdot 
    (\delta_{ik} - \delta_{jk}) \, 
    [\psi(r_{ij}) \mathbf r_{ij} \mathbf r_{ij}
    + \phi(r_{ij}) \mathbf I_{ij}]
\\
  &= 2 (\delta_{ik} - \delta_{jk}) 
      \sum_{(i, j)} [\psi(r_{ij}) (\mathbf g_{ij} \cdot \mathbf r_{ij}) \mathbf r_{ij}
    + \phi(r_{ij}) \mathbf g_{ij}]
\end{align*}



\subsubsection{$\mathbf s = 2 \mathbf r \cdot \nabla \nabla U$}

\begin{align*}
\mathbf s_k &= 2 \mathbf r \cdot \nabla \nabla U
  = 2 \sum_l \mathbf r_l \cdot \nabla_k \nabla_l U
  = 2 \sum_{(i, j)} \sum_l \mathbf x_l \cdot \nabla_k \nabla_l u(r_{ij})
\\
  &= 2 \sum_{(i, j)} \sum_l \mathbf r_l \cdot \nabla_k 
    (\delta_{il} - \delta_{jl}) \, \phi(r_{ij}) \mathbf r_{ij} 
  = 2 \sum_{(i, j)} \mathbf r_{ij} \cdot \nabla_k 
    \, \phi(r_{ij}) \mathbf r_{ij}
\\
  &= 2 \sum_{(i, j)} \mathbf r_{ij} \cdot 
    (\delta_{ik} - \delta_{jk}) \, 
    [\psi(r_{ij}) \mathbf r_{ij} \mathbf r_{ij}
    + \phi(r_{ij}) \mathbf I_{ij}]
\\
  &= 2 (\delta_{ik} - \delta_{jk}) 
      \sum_{(i, j)} [\psi(r_{ij}) r_{ij}^2 + \phi(r_{ij}) ] \mathbf r_{ij}
\end{align*}



\subsubsection{$M = 2 \mathbf g \mathbf g : \nabla \nabla U = \mathbf g \cdot \mathbf h$}

\begin{align*}
M &= \mathbf g \cdot \mathbf h
\\
  &= 2 \sum_{(i, j)} \sum_k (\delta_{ik} - \delta_{jk}) 
    \mathbf g_k \cdot [\psi(r_{ij}) (\mathbf g_{ij} \cdot \mathbf r_{ij}) \mathbf r_{ij}
    + \phi(r_{ij}) \mathbf g_{ij}]
\\
  &= 2 \sum_{(i, j)} 
    \mathbf g_{ij} \cdot [\psi(r_{ij}) (\mathbf g_{ij} \cdot \mathbf r_{ij}) \mathbf r_{ij}
    + \phi(r_{ij}) \mathbf g_{ij}]
\\
  &= 2 \sum_{(i, j)} 
    [\psi(r_{ij}) (\mathbf r_{ij} \cdot \mathbf g_{ij})^2 + \phi(r_ij)\, \mathbf g_{ij}^2].
\end{align*}



\subsubsection{$N = \mathbf g \mathbf g : \nabla \nabla \nabla U \cdot \mathbf g$}

\begin{align*}
N =& \mathbf g \mathbf g : \nabla \nabla \nabla U \cdot \mathbf g
  = \sum_{k, l, m} \mathbf g_k \mathbf g_l : \nabla_k \nabla_l \nabla_m U \cdot \mathbf g_m
\\
  =& \sum_{(i, j)} \sum_{k, l, m} \mathbf g_k \mathbf g_l : \nabla_k \nabla_l \nabla_m u(r_{ij}) \cdot \mathbf g_m
\\
  =& \sum_{(i, j)} \sum_{k, l, m} \mathbf g_k \mathbf g_l : [\nabla_k \nabla_l
    (\delta_{im} - \delta_{jm}) \, \phi(r_{ij}) \mathbf r_{ij}] \cdot \mathbf g_m
\\
  =& \sum_{(i, j)} \sum_{k, l} \mathbf g_k \mathbf g_l : [\nabla_k \nabla_l 
    \, \phi(r_{ij}) \mathbf r_{ij}] \cdot \mathbf g_{ij}
\\
  =& \sum_{(i, j)} \sum_{k, l} \mathbf g_k \mathbf g_l : 
    (\delta_{il} - \delta_{jl}) \, 
    \{
    \nabla [\psi(r_{ij}) \mathbf r_{ij} \mathbf r_{ij}
    + \phi(r_{ij}) \mathbf I_{ij}] 
    \}
  \cdot \mathbf g_{ij}
\\
  =& \sum_{(i, j)} \sum_k \mathbf g_k \mathbf g_{ij} : 
    \{
    \nabla_k [\psi(r_{ij}) \mathbf r_{ij} \mathbf r_{ij}
    + \phi(r_{ij}) \mathbf I_{ij}] 
    \}
  \cdot \mathbf g_{ij}
\\
  =& \sum_{(i, j)} \sum_k \mathbf g_k \mathbf g_{ij} : 
    \{
    (\delta_{ik} - \delta_{jk}) 
    [
    \xi(r_{ij}) \mathbf r_{ij} \mathbf r_{ij} \mathbf r_{ij}
   + \psi(r_{ij}) (\mathbf I_{ij} \mathbf r_{ij}
    + \mathbf I_{ij}^{(13)} \mathbf r_{ij}
    + \mathbf r_{ij} \mathbf I_{ij} ) ] 
    \}
  \cdot \mathbf g_{ij}
\\
  =& \sum_{(i, j)} \mathbf g_{ij} \mathbf g_{ij} : 
    [
    \xi(r_{ij}) \mathbf r_{ij} \mathbf r_{ij} \mathbf r_{ij}
    + \psi(r_{ij}) (\mathbf I_{ij} \mathbf r_{ij}
    + \mathbf I_{ij}^{(13)} \mathbf r_{ij}
    +  \mathbf r_{ij} \mathbf I_{ij})] 
  \cdot \mathbf g_{ij}
\\
  =& \sum_{(i, j)} [
    \xi(r_{ij}) (\mathbf g_{ij} \cdot \mathbf r_{ij})^2
    + 3 \, \psi(r_{ij}) (\mathbf g_{ij} \cdot \mathbf g_{ij})] (\mathbf g_{ij} \cdot \mathbf r_{ij}) 
\end{align*}

\end{document}
