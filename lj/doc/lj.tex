\documentclass{article}
\usepackage{amsmath}
\begin{document}

\section{Switched Lennard-Jones potential}

The potential $u(r)$ is switched from $r = r_s$ to zero $r = r_c$.
\[
  u(r) = \sum_4^7 a_k (r - r_c)^k.
\]
At $r = r_c$, $u = u' = u'' = u''' = 0$;
at $r = r_s$, $u$, $u'$, $u''$, $u'''$ are the same as those at
\[
  u_{LJ}(r) = 4 (r^{-12} - r^{-6}).
\]
The coefficients $a_4, a_5, a_6, a_7$ are 
\begin{align}
  a_4 &= 4 (35 r_s^3 A + 90 r_s^2 \Delta r B + 15 r_s \Delta r^2 C + 28 \Delta r^3 D)/(\Delta r^4 r_s^15), \\
  a_5 &= -24 (14 r_s^3 A + 39 r_s^2 \Delta r B + 7 r_s \Delta r^2 C + 14 \Delta r^3 D)/(\Delta r^5 r_s^15), \\
  a_6 &= 4 (70 r_s^3 A + 20 r_s^2 \Delta r B + 39 r_s \Delta r^2 C + 84 \Delta r^3 D)/(\Delta r^6 r_s^15), \\
  a_7 & = -16 (5 r_s^3 A + 15 r_s^2 \Delta r B + 3 r_s \Delta r^2 C + 7 \Delta r^3 D)/(\Delta r^7 r_s^15),
\end{align}
where $\Delta r = r_c - r_s$, 
$A = 1 - r_s^6$,
$B = 2 - r_s^6$,
$C = 26 - 7 r_s^6$,
$D = 13 - 2 r_s^6$.

\end{document}
